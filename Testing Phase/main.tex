\documentclass[12pt,a4paper,titlepage]{article}
\usepackage[latin1]{inputenc}
\usepackage{amsmath}
\usepackage{amsfonts}
\usepackage{amssymb}
\usepackage{graphicx}

\author{\textit{Broadsword Integration} \\\\
				Marthinus Hermann, 15081479\\
				Regan Koopmans, 15043143 \\
				Jan-Justin van Tonder, 15073298}
\title{\fontsize{40}{40} Integration Testing Report}
\begin{document}
\maketitle

\section{Summary of Testing Methods}

\begin{itemize}
	\item We wrote three bash scripts that drove the majority of the performance testing. We combined these scripts in different orders to primarily generate:
	   \begin{itemize}
			 \item Average response times for particular \texttt{HTTP} requests, performed using the curl utility.
		   \item Load testing, in which we distributed the scripts across physical computers to attempt to emulate a typical usage environment. For this we used the majority of the SIT lab in the Informatorium.
	   \end{itemize}
	\item We used the networking utility \texttt{nmap} to determine open ports and other network vulnerabilities.
	\item We used the software Hydra to attempt to brute-force their \texttt{SSH} password.
\end{itemize}

\section{Testing Results}
\subsection{Functional Requirements}

Testing the functional requirements for integration is a challenging feat. This
is due to the fact that, technically, the only functional requirement for
integration is integration itself. Moreover, not all of the \texttt{GladiOS} submodules
implemented all that was required of them to implement, therefore, there are a
very limited number of chains of requests and responses that can be tested. As
such, from an integration perspective, therefore, no comprehensive integration
test can be performed. Lastly, determining a measure of integration success is
also challenging as there either exists a means of integration and inter-module
communication or there does not.
\\\\
With the aforementioned in mind, we still attempted to test the only functional
requirement and devised a rudimentary test script that performed a total number
of requests to the \texttt{GladiOS} server, logged the responses and finally tallied up
the number of responses for comparison with the number of requests sent. The
idea is that integration involves ensuring that submodules are capable of
communicating with one another, and to this end we attempted to send a request
and see if we received a response back as this meant that the \texttt{GladiOS} server
successfully provided a means for modules to communicate with one another.
Again, this is supposed to be evident by the fact that a request was sent to
the \texttt{GladiOS} server and a response was subsequently received.
\\\\
As for the actual test that was performed, the \texttt{GladiOS} server was the recipient
of 10 000 \texttt{HTTP} requests and responded, on all occasions, with the exact same
number of responses everytime the script was executed. As a result, we graded
the \texttt{GladiOS} functional requirement of integration 10 points (out of 10).

\end{document}
